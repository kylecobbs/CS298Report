\chapter{Conclusion and Future Work}

From the results obtained in this experiment it is evident that the malware detectors of today are incapable of handling code obfuscation in android malware. This problem presents a huge gap in the domain of Android Anit-Virus products.

\section{Conclusion}

Due to the limited processing capacity of the mobile devices, it is imperative that stand alone malware detectors are able to sufficiently defend against known threats and variants of known malware. 

The current generation of malware detectors are incapable of handling encryption in the body of malware. This experiment reiterates this fact and supports the conclusions drawn by Preda et al. in \cite{aamo}. The conclusions drawn by them indicating a huge gap in the requirement and the availability of sophisticated anti-virus products is still very much prevalent.

\section{Future Work}

Similar to the obfuscators employed in this experiment, it should be possible to generate "de-obfuscators" for Android files. The ease of decompiling and compiling APK files makes it an easy target for malware writers.

If access to the source code of antivirus products are provided, defense mechanisms against such obfuscation techniques can be built in. With our increasing dependence on mobile phones and their proliferation into our lives, it is of utmost importance that sophisticated malware detectors are able to handle obfuscated malware. 

